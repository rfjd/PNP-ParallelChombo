\documentclass{article}
\usepackage[margin=1in]{geometry}
\usepackage{enumitem}
\usepackage{amsmath}
\usepackage{hyperref}
\hypersetup{
	colorlinks=true,
%	linkcolor=blue,
%	filecolor=magenta,      
%	urlcolor=blue,
}
\renewcommand\UrlFont{\color{blue}\rmfamily\itshape}

\title{\textbf{Steps to set up mpich, hdf5, \& Chombo}}
\author{\textbf{Aref Hashemi}}
\date{\today}

\begin{document}
\maketitle

\begin{enumerate}[label=\textbf{\arabic*})]
\item \textbf{apt-get install X}

  \textbf{X} $\equiv$ gfortran, gnuplot, doxygen, graphviz, libblas-dev, liblapack-dev, libgtk2.0-dev, tcl, tcl-dev, m4, csh, subversion.
  
\item \textbf{source.} autoconf, automake, libtool

  \begin{itemize}
  \item .\slash configure -{}-prefix=\slash usr\slash local
  \item make
  \item sudo make install
  \end{itemize}

\item \textbf{mpich}
  
  \begin{itemize}
  \item .\slash configure -{}-prefix=\slash usr\slash local\slash PROG\slash mpich
  \item make
  \item sudo make install. Disregard the possible warnings about the libraries.
  \end{itemize}
  
\item \textbf{hdf5-serial}

  \begin{itemize} 
  \item .\slash configure -{}-prefix=\slash usr\slash local\slash PROG\slash hdf5-serial -{}-enable-fortran
  \item make
  \item make check
  \item sudo make install
  \end{itemize}


\item \textbf{hdf5-parallel}

  \begin{itemize}
  \item CC=\slash usr\slash local\slash PROG\slash mpich\slash bin\slash mpicc .\slash configure -{}-prefix=\slash usr\slash local\slash PROG\slash hdf5-parallel -{}-enable-fortran -{}-enable-parallel
  \item make
  \item make check
  \item sudo make install.
  \end{itemize}

\item Add the bashrcadd content to the end of the $\sim$\slash .bashrc file.
  
\item \textbf{Chombo}
  \begin{itemize}
  \item register in \href{https://commons.lbl.gov/display/chombo/Chombo+Download+Page}{Chombo Download Page} to get USERNAME \& PASSWORD
  \item svn -{}-username USERNAME co https:\slash \slash anag-repo.lbl.gov\slash svn\slash Chombo\slash release\slash \{ver\} \{Directory-PATH-to-Chombo\}. See \href{https://anag-repo.lbl.gov/chombo-3.2/access.html}{Download Instructions} for more information.

    e.g., for Chombo version 3.2, USERNAME=rfjd, download directory \slash home\slash rfjd\slash Chombo:\\svn -{}-username rfjd co https:\slash \slash anag-repo.lbl.gov\slash svn\slash Chombo\slash release\slash 3.2 \slash home\slash rfjd\slash Chombo\slash
  \item copy the file Make.defs.local into \{Directory-PATH-to-Chombo\}\slash lib\slash mk
  \item cd \{Directory-PATH-to-Chombo\}\slash lib
  \item make lib
  \item make doxygen. Chombo documentation and info can be found in \{Directory-PATH-to-Chombo\}\slash lib\slash doc\slash doxygen\slash html\slash index.html using your web browser.
  \item To run the released examples, go to the corresponding directory, make all, mpiexec .\slash X.ex inputs. You will be able to open *.hdf5 files using visit if DIM = 2 or 3 in Make.defs.local.
  \end{itemize}

\renewcommand{\_}{\texttt{\detokenize{_}}}
\item \textbf{visit}
  \begin{itemize}
  \item download the visit install script (visit\_installX\_Y\_Z, where X.Y.Z is the version) and platform from the \href{https://wci.llnl.gov/simulation/computer-codes/visit/executables}{official download page}.
  \item chmod +x visit-installX\_Y\_Z
  \item ./visit-installX\_Y\_Z X.Y.Z platform /usr/local/visit

  e.g., for linux-x86\_64-ubuntu20 (platform) and version 3.1.4:\\
  ./visit-install3\_1\_4 3.1.4 linux-x86\_64-ubuntu20 /usr/local/visit  
  \item choose some number in the pop-up!
  \item add {alias visit='/usr/local/visit/bin/visit'} to the .bashrc file.
  \item typing `visit' in a terminal brings up the software! You can then load the hdf5 files to view the Chombo results.
  \end{itemize}
  
\end{enumerate}

The above procedure was checked with Ubuntu 20.04 LTS, \href{https://ftp.gnu.org/gnu/autoconf/}{autoconf-2.71}, \href{https://ftp.gnu.org/gnu/automake/}{automake-1.16.5}, \href{https://mirrors.ocf.berkeley.edu/gnu/libtool/}{libtool-2.4.7}, \href{https://wci.llnl.gov/simulation/computer-codes/visit/executables}{visit3\_1\_4.linux-x86\_64-ubuntu20}, \href{https://www.mpich.org/downloads/}{mpich-4.1.1}, and \href{https://www.hdfgroup.org/downloads/hdf5/source-code/}{hdf5-1.14.0}.
\end{document}
